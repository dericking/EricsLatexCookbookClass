%====================================================================
% Recipe include file template (example recipe)
%--------------------------------------------------------------------
% Save as: recipes/roasted_garlic_lemon_pasta.tex
% Include from main.tex with:
%   \include{recipes/roasted_garlic_lemon_pasta}
%
% This file is NOT standalone. Do not add \documentclass, \usepackage,
% or \begin{document}. Those live in main.tex / recipebook.cls.
%====================================================================

%--------------------------------------------------------------------
% \begin{recipe}{<Recipe Title>}{<Recipe Author>}
%
% Sets the page header (title left; author right) and prints the title block.
%--------------------------------------------------------------------
\begin{recipe}{Example: Roasted Garlic Lemon Pasta}{Generated Example}

%--------------------------------------------------------------------
% \begin{RecipeDescription}[<image>][<left width>][<draft note>] ... \end{RecipeDescription}
%
% Optional arguments:
%   [<image>]      If present, shows image in left mini-column.
%   [<left width>] Overrides the left mini-column width for the description.
%   [<draft note>] If non-empty, overlays an FPO badge on the image and prints
%                  the note under the image (use for placeholder/rights issues).
%
% If you omit [<image>], the description spans full width and the other optional
% args are ignored.
%--------------------------------------------------------------------
\begin{RecipeDescription}
Generated example. A pasta with roasted garlic, lemon, and parmesan.

%--------------------------------------------------------------------
% \EstimatedNutrition{<servings>}{<cal/serving>}{<protein g>}{<carbs g>}{<fat g>}
%
% Prints a single italic line under the description.
% Prints nothing if ALL five args are blank.
%--------------------------------------------------------------------
\EstimatedNutrition{2}{620}{22}{78}{24}
\end{RecipeDescription}

% Optional manual spacing (you can remove this and tune class spacing instead).
\bigskip

%--------------------------------------------------------------------
% \begin{recipecols} ... \end{recipecols}
%
% Two-column layout:
%   left  = Ingredients
%   right = Procedure
%
% Must use \RecipeColumnBreak to switch columns.
%--------------------------------------------------------------------
\begin{recipecols}

%--------------------------------------------------------------------
% Ingredients: itemize list; bullets align with the header.
%--------------------------------------------------------------------
\begin{Ingredients}
\item 8 oz pasta (spaghetti, linguine, or penne)
\item 1 head garlic
\item 2 Tbsp olive oil (plus more for roasting)
\item 1 lemon (zest + juice)
\item 1/4 tsp salt (plus more for pasta water)
\item Black pepper
\item 1/3 cup grated parmesan (or pecorino)
\item 1/4 cup pasta water (reserved; more as needed)
\item Optional: pinch of red pepper flakes
\item Optional: chopped parsley
\end{Ingredients}

\RecipeColumnBreak

%--------------------------------------------------------------------
% Procedure: enumerate list; numbers align with the header.
% Keep steps short and direct; add details as needed.
%--------------------------------------------------------------------
\begin{Procedure}
\item Preheat oven to 400\textdegree F.
Slice the top off the garlic head, drizzle with olive oil, wrap in foil, and roast 35--45 minutes until soft.
\item Cook pasta in well-salted water until al dente.
Reserve at least 1/2 cup pasta water, then drain.
\item Squeeze roasted garlic cloves into a bowl and mash into a paste.
Stir in 2 Tbsp olive oil, lemon zest, lemon juice, salt, and pepper (and red pepper flakes if using).
\item Toss pasta with the garlic-lemon mixture.
Add reserved pasta water a splash at a time until glossy and lightly sauced.
\item Add parmesan and toss again.
Taste and adjust with more salt, pepper, lemon, or cheese.
\item Serve immediately.
Top with parsley and extra parmesan if desired.
\end{Procedure}

\end{recipecols}

%--------------------------------------------------------------------
% Notes: plain text by default (no bullets).
% Add your own itemize/enumerate here if you want structured notes.
%--------------------------------------------------------------------
\begin{Notes}
If you want extra protein, add a can of white beans or shredded rotisserie chicken when you toss the pasta.
Roasted garlic keeps in the fridge for a few days; you can roast two heads and use one later.
\end{Notes}

\end{recipe}
